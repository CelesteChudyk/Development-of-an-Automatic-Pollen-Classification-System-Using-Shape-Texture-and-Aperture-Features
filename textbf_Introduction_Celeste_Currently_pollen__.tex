\textbf{Introduction} (Celeste)
     
     Currently, pollen count information is usually limited to generalizing all pollen types with no access to information regarding particular species. In order to differentiate species, typically a trained palynologist would have to manually count samples using a microscope. Advances in image processing and machine learning enable the development of an automatic system that, given a digital image from a lightfield microscope, can automatically detect and describe the species of pollen particles present. 
     
We build upon previous work from within our lab which has planned the structure for a complete personal pollen tracker [Lozano,etc], as well as started the research for such an image classification task. Preliminary results have shown that extraction of both shape features and aperture features lead to useful results [Lozano,etc]. To expand on this research, we have built a software process that automatically includes such features for consideration together, as well as added texture features. Moreover, the range of tested data has been greatly expanded in order to avoid model overfitting. 

    
\section{Classification}

Once the shape, texture, and aperture features have been calculated, they are added together into a csv file. A data set of 5 species with 40 sample pollen images from 3 separate sample slides led to a total of 600 samples, each with 252 extracted features. A supervised learning process used this data for model creation, which was then tested using ten-folds cross validation. Both support vector machines and a random forest classifier showed promising (and very similar results); for the results reported here a random forest classifier was used due to faster processing on the larger data sets. The n-estimators parameter for this method was set to a typical size of 100 (increasing this number did lead to slightly improved results yet also dramatically increased processing times).
    
  
  
  
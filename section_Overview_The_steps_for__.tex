\section{Overview} 

The steps for our process are as follows: 1. Image acquisition and particle segmentation, 2. Feature extraction, and 3. Classification.

Our process begins with scanning glass slides of the various pollen species with a digital microscope, then segmenting these images to gather samples of individual pollen particles. These images are then further segmented to identify the pollen boundary, and the area within this boundary is used for feature extraction. 18 shape features, texture features including the Fast Fourier Transform, Local Binary Patterns, Histogram of Oriented Gradients, and Haralick features, as well as aperture features are used. These features are then trained using supervised learning to build a model for the 5 pollen species sampled. The model is then tested with ten-fold cross validation. The process is illustrated in figure \ref{fig:process}. 
    
    
    
    
    
    
    
    
    
    
    
    
  
  
  
  
  
  
  
  
  
  
  
  
  
  
  
  
  

\subsubsection{Haralick Features}

Haralick features are determined by computations over the GLCM (Grey Level Co-Occurance Matrix): the angular second moment, contrast, correlation, sum of squares: variance, inverse difference moment, sum average, sum variance, sum entropy, entropy, difference variance, difference entropy, measure of correlation 1, and measure of correlation 2. These are 13 out of the 14 original features developed by Haralick: the 14th is typically left out computations due to uncertainty in the metric's stability. 

\subsubsection{Histogram oriented gradient (HOG)}

The Histogram of Oriented Gradients is calculated by first determining gradient values over a 3 by 3 Sobel mask. Next, bins are created for the creation of cell histograms; here, 10 bins were used. The gradient angles are divided into these bins, and the gradient magnitudes of the pixel values are used to determine orientation. After normalization, the values are flattened into one feature vector. 

\subsubsection{Local Binary Pattern (LBP)}

To obtain local binary patterns, a 3 by 3 pixel window is moved over the image, and the value of the central pixel is compared to the value of its neighbors. In the case that the neighbor is of lower value, it is assigned a zero, and in the case of a higher value, a one. This string of eight numbers ("00011101" for instance) is the determined local pattern. Frequences of particular patterns can be used as the texture description. 
  
  

  
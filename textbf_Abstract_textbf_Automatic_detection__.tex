\textbf{Abstract\textbf{}}


Automatic detection and classification of pollen species has value for use inside of palynologic allergen studies. Traditional labeling of different pollen species requires an expert biologist to classify particles by sight, and is therefore time-consuming and expensive. Here, an automatic process is developed which segments the particle contour and uses the extracted features for the classification process. We consider shape features, texture features and aperture features and analyze which are useful. The texture features analyzed include: Gabor Wavelets, Fast Fourier Transform, Local Binary Patterns, Hough of Gradients, and Haralick features. We have streamlined the process into one code base, and developed multithreading functionality to decrease the processing time for large datasets.
    
    
    
    
    
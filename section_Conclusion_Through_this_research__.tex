\section{Conclusion}

Through this research, we have tested an expanded sample set of 5 species of pollen particles and used shape, texture and aperture features for use in classification. Use of all features led to an accuracy of $87\% \pm 2\%$. Through testing of individual texture features in combination with shape features, it was found that using only the shape and Haralick features resulted in an accuracy of $87\% \pm 3\%$. Gabor Filters also proved to be a useful feature as seen through the improved accuracy compared to using just the shape features alone. Surprisingly, the other texture features as well as the aperture features did not result in significant accuracy gains. One next step of research would be to investigate under which exact conditions certain texture features prove useful. In the case of the aperture features, one known limitation is that the aperture types were trained on a more limited dataset. Because the aperture detection process technique developed did have positive results in determining correct aperture positions, it would be interesting to retrain the aperture type on a wider dataset and see if this results in a more useful set of extracted features. Future research would also include application of this process to data collected outside of a laboratory environment, as well as expansion to include more pollen species.
    
  
  
  
  
  
\textbf{Structure of Program/Software overview} (Hugo) 

The system is made of two principal steps which are called MODELER and MATCHER those steps must be executed once after an other to let the application works properly.

The \textbf{MODELER} is very easy to perform, the user have to create it's “type” folders signed by an underscore.
Each folder should have a name related to the type of pollen that it is containing. For instance, the folder “\_Mugwort” should contain only Mugwort pollen type particles.

Once the MODELER directory have been setup, the user will be able to launch the model generator by double clicking on the EXE – MODELER script.
Finally, when the model generator script is finished, a file called "model.csv" is created in the back directory. This file contains (by rows) all the particles present in the underscored folders. The particles are represented by their type number (first column) and their attributes (other columns). Also, a mosin.csv file (standing for Model Signification) is created. It contains the type identification informations to help the matcher knowing which type is referred by which number.

The \textbf{MATCHER} can now be used. It is the most important step of the system, this is where the application will try to find for each user's particles its type.
First, the user will have to take pictures of its pollen particles using a microscope and copy those pictures into the “Your images” folder inside the MATCHER one.

This part of the program works using three sub-scripts.

\begin{itemize}
\item The EXE\_SEGMENTER which will cut off particles from the microscope images and save them in the toClass folder

\item the EXE\_CLASSIFIER which will fit those saved particles with the model and try to determine their types in a sorted.csv file.

\item And finally the EXE\_MOSIN which goal is to transform the saved particle images in the toClass folder with their types name by using the mosin.csv generated with the model and sorted.csv file.
\end{itemize}
    
    
    
    
    
    
    
    
    
    
    
    
    
    
    
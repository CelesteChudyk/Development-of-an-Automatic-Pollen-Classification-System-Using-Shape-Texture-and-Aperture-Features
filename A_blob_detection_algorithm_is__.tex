
A blob detection algorithm is now applied in order to extract a small image surrounding each particle. This algorithm is based on four attributes – Area, Circularity, Convexity and Inertia Ration, with parameters for “minimum” and “maximum” values for each. By setting the parameters for the expected characteristics of pollen grains, the smaller images are then found and extracted.


One last filter is then used on the resulting images of particles [Figure 3]. Because the pollen grains settle into the slide adhesive at different depths, some particles will be out of focus. These blurry images will provide insufficient data especially concerning texture features, therefore we remove them from our analysis. A blur detection algorithm was developed and applied to each image: a Laplacian filter set to a manually determined threshold value determines which images are too blurry and removed from further processing steps. 

Lastly, the contour surrounding each pollen particle is identified, using OpenCV's \verb|findContours()| method. 

  
  
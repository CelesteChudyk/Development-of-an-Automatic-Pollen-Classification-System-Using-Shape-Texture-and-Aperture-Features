\subsubsection{Aperture Detection}

The number and type of apertures present on the pollen surface is a typical feature used by palynologists in order to determine the pollen species. Therefore, it seems useful to also build an automatic aperture detection function in order to identify and count apertures as an addition feature set. Preliminary work identifying apertures (Lozano) has shown potential for this analysis.
First, a moving window segments the pollen image into smaller areas. Each smaller image is manually labeled as an aperture or not an aperture. Texture features are extracted from these smaller images, including those through a Fast Fourier Transform (FFT), Gabor Filters (GF), Local Binary Pattern (LBP), Histogram of Oriented Gradients (HOG), and Haralick features. A supervised learning process (through the use of support vector machines) then creates a model for each of the four species expected to include apertures on the surface.
Once an unlabeled pollen image is given to be classified, the system again uses a moving window to break up the image into subsections. These smaller sections are then loaded into the generated model, and four values are returned for each detected aperture, corresponding to the probability that the aperture is of type Alder, Birch, Hazel, and Mugwort. 

    
    
  
  
  
  
  
  
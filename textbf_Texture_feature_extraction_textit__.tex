\textbf{Texture feature extraction}


    \textit{Introduction}
    
The texture features extracted included: Gabor Wavelets (GW), the Fast Fourier Transform (FFT), the Local Binary Pattern (LBP), the Gray-Level Co-occurrence Matrix (GLCM), the Histogram of Oriented Gradients (HOG), and Haralick features.

    \textit{Gabor Wavelets}

The Gabor Wavelets function consists of the application of 5 different size masks and 8 orientation masks in order to produce output images. For each of the 40 resulting images, we calculate the local energy over the entire image (the sum of the square of the gray-level pixel intensity), and the mean amplitude (the sum of the amplitudes divided by the total number of images). In addition to these 80 values, we also store the total local energy for each of the 8 directions as well as the direction where the local energy is at the maximum.

    \textit{Fourier Transform}

We apply a Fast Fourier Transform to the image, apply a logarithmic transformation, and create a graph of the resulting frequency domain. After taking the highest 10 frequency peaks, we compute the differences between the peaks and store these values, as well as the mean of the differences and the variance of the differences. 

    
    
    
    
    
  
  